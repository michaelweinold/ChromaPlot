\documentclass[11pt]{scrbook}

% Must be loaded before subfiles package
\usepackage{../supplementary/header/header_chromaplot_main}
\usepackage{../supplementary/header/header_chromaplot_plotting}
\usepackage{../supplementary/header/header_chromaplot_supplementary}
\usepackage{../glossary/header_chromaplot_glossary}

\usepackage{subfiles} % https://ctan.org/pkg/subfiles

% Must be loaded after subfiles package
\usepackage{../bibliography/header_chromaplot_bibliography}

\begin{document}
%%%%%%%%%%%%%%%%%%%%%%%%%%%%%%%%%%%%%%%%%%%%%%%%%%%%%%%%%%%%%%%%%%

\section{Introduction}

    \subsection{The Need for High Quality Chromaticity Diagrams}

        Even highly cited articles in the world's most recognizable scientific journals \cite{sun2007bright} or specialist publications \cite{mckittrick2014down} include low-quality, often pixelated or scanned chromaticity diagrams. Patents and technical documentation often omit colors all together, instead opting for a grid in combination with coordinate axes \cite{shimizu1999light}.
        A good example is \cite{doi2012led}. Examples of publications using low-quality rasterized renderings include university lecture documents \cite{univie2020chromaticity}.

The first documented attempt to create a vectorized chromaticity diagram with \LaTeX packages identified by the authors was in a \textit{StackExchange} question and subsequent answer by Paolo Brasolin in 2014 \cite{brasolin2014stackexchange}.

\subfile{../glossary/glossary}

\printbibliography

%%%%%%%%%%%%%%%%%%%%%%%%%%%%%%%%%%%%%%%%%%%%%%%%%%%%%%%%%%%%%%%%%%
\end{document}
